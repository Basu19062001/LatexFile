\documentclass[12pt,a4paper,oneside]{report}
%\documentclass[12pt,a4paper,twoside]{book}
\usepackage{chngcntr}
%\counterwithin{figure}{chapter}
%\usepackage[nottoc]{tocbibind}
\usepackage{changepage}
\pagestyle{plain}
\usepackage{makeidx}
\usepackage{nomencl}
\usepackage{titlesec}
\usepackage{epsf}
\usepackage{graphicx}
\usepackage{amsmath}
\usepackage{amsfonts}
\usepackage{epstopdf}
\usepackage{comment}
\usepackage[hidelinks]{hyperref}
\setcounter{secnumdepth}{3}
\setcounter{tocdepth}{3}
\usepackage{fancyhdr}
\usepackage[english]{babel}
\usepackage{blindtext}
\usepackage{upquote}
%\usepackage[titletoc]{appendix}
\usepackage{enumerate}
\usepackage{lipsum}
\usepackage[a4paper, inner=1.5cm, outer=3cm, top=2cm,bottom=3cm, bindingoffset=1cm]{geometry}
\textheight=235mm
\textwidth=145mm
\topmargin=0mm
\headheight=0mm
\headsep=0mm
\oddsidemargin=15mm
\evensidemargin=15mm
\makeindex
\makeglossary
\graphicspath{{./figures/}}
\usepackage{float}
\usepackage{microtype}
\usepackage[]{setspace}
\singlespacing
\usepackage{amssymb}
\setstretch{1}
\usepackage{cite}
\linespread{1.3}
\renewcommand{\thesection}{\arabic{section}}
\usepackage[english]{babel}
\addto{\captionsenglish}{%
	\renewcommand{\bibname}{\LARGE References}
}

%***********************************************************

\begin{document}
	\newgeometry{top=0.5in,bottom=1in,right=1in,left=1in}
	\thispagestyle{empty}
	\begin{center}
		{\textbf{\textit{Synopsis of}\\}}
		{\Large\bf{Fetal Movement Detection [IOT]} \\}
		
		
		\vspace{0.3in}
		
		
		{\textbf{\textit  {A Project to be submitted in partial fulfillment by\\}}}
		
		% \vspace{0.3 in}
		
		%       {\bf\textit{to be submitted by}}
		
		\vspace{0.4in}
		\textbf{\Large Avishek Ray [TL]
			\\ \small Univ.Roll No. 15500121076
			\\ \Large Ajoy Kumar Dutta
			\\ \small Univ.Roll No. 15500121066
			\\ \Large Basudev Samanta
			\\ \small Univ.Roll No. 15500121067\\}
		
		
		\vspace{0.3 in}
		
		{\bf\textit{for the award of the degree of}}
		
		\vspace {0.3 in}
		
		% {\bf\textit{of}}
		
		%  \vspace{0.3in}
		{\Large{\bf B.Tech \\ \small in \\\Large Computer Science and Engineering }\\}
		\vspace{0.3in}
		{\bf\textit{Under the supervision of\\ }}
		{\Large{\bf Prof. Subhraprakash Dutta}\\}    
		\vspace{0.3in}
		\includegraphics[width=0.30\linewidth]{../../CollegeLogo}
		
		
		%\vspace{0.2in}
		{\large\bf Department of Computer Science\\}
		
		{\large\bf  Durgapur Institute of Advanced Technology and Management \\}
		{\large\bf Rajbandh, Durgapur \\}
		{\large\bf  2023 }
	\end{center}
	\date{}
	\setcounter{page}{0}
	
	\newpage
	\tableofcontents
	
	% \newpage
	%\listoffigures
	
	\newpage
	\section*{Abstract}
		The World Health Organization (WHO) raised concerns about the significant risks mothers face during pregnancy and childbirth in India. Of particular note was the alarming rate of emergency postpartum hysterectomies, reaching 83 per 100,000 cases. Additionally, in low- and middle-income countries (LMICs), adolescents aged 15–19 experienced approximately 21 million pregnancies annually, with half of them unintended, resulting in around 12 million births.		
		To tackle these issues, a solution has been proposed – a special belt with AI and IoT technologies. This belt continuously checks vital health signs of pregnant women, analyzes the data, and sends it to ThingSpeak, an IoT cloud platform. The development includes machine learning for fetal heart classification and identifying pregnancy risks, providing important insights for both mother and child's well-being.
		
		One key aspect of this proposed system is the use of sensors such as accelerometers and pulse sensors to monitor fetal movements and heart rate. This addresses challenges associated with the accuracy of fetal movement detection using traditional monitors, difficulties in in-person monitoring by pregnant women, and limitations in monitoring duration with ultrasonic Doppler imaging devices. Almost all women who have delivered a live-born baby, more than 99\% agreed with the statement that it was important to them to feel the baby move every day.
		
		The wearable fetal movement detection system represents a significant advancement in maternal and fetal health monitoring. It introduces a novel approach that recognizes the critical relationship between fetal activity, movement, welfare, kicks and developmental progress. The system comprises accelerometer sensors integrated with an ARDUINO microcontroller and interfaces with MATLAB (Matrix Laboratory) for signal processing.
		
		By focusing on this life course perspective, the proposed solution aligns with the imperative for achieving the Sustainable Development Goals (SDGs) related to maternal and newborn health. This technological innovation not only addresses existing challenges but also paves the way for a comprehensive and data-driven approach to enhancing the health outcomes of both mothers and their newborns, contributing to global efforts to achieve a healthier and more sustainable future.
		
		\subsection*{Keywords:}Accelerometer Sensors, ThingSpeak, ARDUINO, MATLAB, Heart rate Pulse Sensor, IoT cloud platform, Machine learning.
		
	
	
	
	%****************************************************************
	\newpage
	\section{Introduction}
	Maternal health, encompassing the well-being of women during pregnancy, childbirth, and the immediate postpartum period, is a critical aspect of global healthcare. Inadequate treatment and education, particularly in economically disadvantaged countries, contribute to maternal deaths. Hypertension in pregnant women is a prevalent issue worldwide, with conditions like preeclampsia posing significant risks. Routine prenatal appointments play a crucial role in monitoring and managing maternal health, involving aspects such as nutrition and exercise planning, as well as medical assessments like fetal heart rate monitoring, blood pressure checks, weight measurements, fundal height assessments, and urine testing. Regular medical checks and fetal kick count monitoring are essential for early detection of potential health issues in pregnant women.
	
	A significant challenge, especially in rural areas, lies in the lack of awareness about the importance of proper medication and the financial constraints preventing access to necessary medical care. While ultrasonic scanning devices are available, their high costs pose a barrier to widespread use.
	
	The integration of Internet of Things (IoT) technology offers a promising solution to address these challenges. By leveraging embedded devices, communication protocols, sensor networks, internet protocols, and applications, IoT transforms conventional healthcare tools into intelligent, connected systems. This enables the seamless collection, management, and sharing of all healthcare-related information, encompassing diagnosis, treatment, recovery, inventory, and medication.
	
	The proposed system employs various sensors, including a heart rate pulse sensor and an accelerometer sensor to monitor fetal kicks. The data collected from these sensors is communicated through an Arduino Uno and IoT technology to a software program. This software can be loaded onto a mobile device or PC, allowing healthcare providers and expecting mothers to access graphical data through an LCD display. Importantly, the system includes alerts through a buzzer to promptly notify users of any abnormal values, enabling timely intervention and improving maternal health outcomes. This innovative approach not only addresses the challenges faced in maternal healthcare but also highlights the potential of technology, specifically IoT, in revolutionizing traditional healthcare practices for better accessibility and effectiveness.\cite{b1}
	
	Fetal monitoring is a crucial aspect of prenatal care, and ultrasound methods are predominant in clinical practice. Among these, fetal echocardiography (fECHO) is commonly employed between the 20th and 23rd weeks of pregnancy for diagnosing congenital heart defects. Additionally, cardiotocography (CTG) is widely used to simultaneously measure the fetal heartbeat and maternal uterine contractions, contributing to a reduction in newborn mortality rates during delivery.
	
	While CTG has proven effective, its drawback lies in high sensitivity to various types of noise caused by maternal movements. This sensitivity necessitates frequent repositioning of the ultrasound transducers, which can be inconvenient and impact the accuracy of the monitoring process.
	
	Alternative methods for fetal heart rate monitoring include fetal electrocardiography (fECG), fetal phonocardiography (fPCG), and fetal magnetocardiography (fMCG). Fetal electrocardiography provides insight into the electrical activity of the fetal heart, while fetal phonocardiography captures its mechanical (acoustic) activity. Fetal magnetocardiography involves recording the magnetic field produced by the fetal heart.
	
	Each of these methods offers distinct advantages and insights into fetal well-being. However, it's crucial to acknowledge the limitations of each technique and consider factors such as sensitivity to noise and the need for repositioning in the overall assessment of their effectiveness in monitoring fetal health. As technology continues to advance, there may be opportunities to address these limitations and further refine fetal monitoring methods for enhanced accuracy and convenience in clinical settings.\cite{b2}
	
	
	
	% A promising  approach is the measurement of ultrasonic velocity
	%**************************************************************
	\section{Motivation}
	\lipsum[2-4]
	
	
	\section{Project Objectives}
	This project investigates......
	\begin{itemize}
		\item objective 1
		\item objective 2
		\item objective 3
		\item objective 4
		\item  objective 5
		
	\end{itemize}
	
	
	
	
	
	%***************************************************************
	\section{Literature Review}
	In this project, we propose methods .................. A brief description of the contributions of this thesis is given below:
	\begin{itemize}
		\item
		
		
	\end{itemize}
	%****************************************************************
	%\section{Conclusions}
	
	%****************************************************************
	\newpage
	\section{Methodology}
	The project consists  of seven  chapters, and the organization of the project is as follows:
	\begin{enumerate}
		\item
		
	\end{enumerate}
	%\pagebreak
	%****************************************************************
	
	\newpage
	\section{Published Article Review}
	\begin{enumerate}
		\item Publication Name 1  
		\item publication 2
		\item publication 3
		\item publication 4
		\item publication 5
		
	\end{enumerate}
	%%%%%%%%%%%%%%%%%%%%%%%%%%%%%%%%%%%%%%%%%%%%%%%%%
	
	\newpage
	\section{Conclusion}
	Write the conclusion of the project.
	\begin{enumerate}
		\item
	\end{enumerate}
	
	\newpage
	\section{Future Work}
	Write the future scopes of the project.
	\begin{enumerate}
		\item
		
	\end{enumerate}
	\bibliographystyle{plain}
	\bibliography{synopsis}
	%\nocite{*}
	
\end{document}
\documentclass[12pt,a4paper,oneside]{report}
%\documentclass[12pt,a4paper,twoside]{book}
\usepackage{chngcntr}
%\counterwithin{figure}{chapter}
%\usepackage[nottoc]{tocbibind}
\usepackage{changepage}
\pagestyle{plain}
\usepackage{makeidx}
\usepackage{nomencl}
\usepackage{titlesec}
\usepackage{epsf}
\usepackage{graphicx}
\usepackage{amsmath}
\usepackage{amsfonts}
\usepackage{epstopdf}
\usepackage{comment}
\usepackage[hidelinks]{hyperref}
\setcounter{secnumdepth}{3}
\setcounter{tocdepth}{3}
\usepackage{fancyhdr}
\usepackage[english]{babel}
\usepackage{blindtext}
\usepackage{upquote}
%\usepackage[titletoc]{appendix}
\usepackage{enumerate}
\usepackage{lipsum}
\usepackage[a4paper, inner=1.5cm, outer=3cm, top=2cm,bottom=3cm, bindingoffset=1cm]{geometry}
\textheight=235mm
\textwidth=145mm
\topmargin=0mm
\headheight=0mm
\headsep=0mm
\oddsidemargin=15mm
\evensidemargin=15mm
\makeindex
\makeglossary
\graphicspath{{./figures/}}
\usepackage{float}
\usepackage{microtype}
\usepackage[]{setspace}
\singlespacing
\usepackage{amssymb}
\setstretch{1}
\usepackage{cite}
\linespread{1.3}
\renewcommand{\thesection}{\arabic{section}}
\usepackage[english]{babel}
\addto{\captionsenglish}{%
	\renewcommand{\bibname}{\LARGE References}
}

%***********************************************************

\begin{document}
	\newgeometry{top=0.5in,bottom=1in,right=1in,left=1in}
	\thispagestyle{empty}
	\begin{center}
		{\textbf{\textit{Synopsis of}\\}}
		{\Large\bf{Fetal Movement Detection [IOT]} \\}
		
		
		\vspace{0.3in}
		
		
		{\textbf{\textit  {A Project to be submitted in partial fulfillment by\\}}}
		
		% \vspace{0.3 in}
		
		%       {\bf\textit{to be submitted by}}
		
		\vspace{0.4in}
		\textbf{\Large Avishek Ray [TL]
			\\ \small Univ.Roll No. 15500121076
			\\ \Large Ajoy Kumar Dutta
			\\ \small Univ.Roll No. 15500121066
			\\ \Large Basudev Samanta
			\\ \small Univ.Roll No. 15500121067\\}
		
		
		\vspace{0.3 in}
		
		{\bf\textit{for the award of the degree of}}
		
		\vspace {0.3 in}
		
		% {\bf\textit{of}}
		
		%  \vspace{0.3in}
		{\Large{\bf B.Tech \\ \small in \\\Large Computer Science and Engineering }\\}
		\vspace{0.3in}
		{\bf\textit{Under the supervision of\\ }}
		{\Large{\bf Prof. Subhraprakash Dutta}\\}    
		\vspace{0.3in}
		\includegraphics[width=0.30\linewidth]{../../CollegeLogo}
		
		
		%\vspace{0.2in}
		{\large\bf Department of Computer Science\\}
		
		{\large\bf  Durgapur Institute of Advanced Technology and Management \\}
		{\large\bf Rajbandh, Durgapur \\}
		{\large\bf  2023 }
	\end{center}
	\date{}
	\setcounter{page}{0}
	
	\newpage
	\tableofcontents
	
	% \newpage
	%\listoffigures
	
	\newpage
	\section*{Abstract}
		The World Health Organization (WHO) raised concerns about the significant risks mothers face during pregnancy and childbirth in India. Of particular note was the alarming rate of emergency postpartum hysterectomies, reaching 83 per 100,000 cases. Additionally, in low- and middle-income countries (LMICs), adolescents aged 15–19 experienced approximately 21 million pregnancies annually, with half of them unintended, resulting in around 12 million births.		
		To tackle these issues, a solution has been proposed – a special belt with AI and IoT technologies. This belt continuously checks vital health signs of pregnant women, analyzes the data, and sends it to ThingSpeak, an IoT cloud platform. The development includes machine learning for fetal heart classification and identifying pregnancy risks, providing important insights for both mother and child's well-being.
		
		One key aspect of this proposed system is the use of sensors such as accelerometers and pulse sensors to monitor fetal movements and heart rate. This addresses challenges associated with the accuracy of fetal movement detection using traditional monitors, difficulties in in-person monitoring by pregnant women, and limitations in monitoring duration with ultrasonic Doppler imaging devices. Almost all women who have delivered a live-born baby, more than 99\% agreed with the statement that it was important to them to feel the baby move every day.
		
		The wearable fetal movement detection system represents a significant advancement in maternal and fetal health monitoring. It introduces a novel approach that recognizes the critical relationship between fetal activity, movement, welfare, kicks and developmental progress. The system comprises accelerometer sensors integrated with an ARDUINO microcontroller and interfaces with MATLAB (Matrix Laboratory) for signal processing.
		
		By focusing on this life course perspective, the proposed solution aligns with the imperative for achieving the Sustainable Development Goals (SDGs) related to maternal and newborn health. This technological innovation not only addresses existing challenges but also paves the way for a comprehensive and data-driven approach to enhancing the health outcomes of both mothers and their newborns, contributing to global efforts to achieve a healthier and more sustainable future.
		
		\subsection*{Keywords:}Accelerometer Sensors, ThingSpeak, ARDUINO, MATLAB, Heart rate Pulse Sensor, IoT cloud platform, Machine learning.
		
	
	
	
	%****************************************************************
	\newpage
	\section{Introduction}
	\lipsum[2-4]\cite{shaoo2020}
	
	
	
	
	
	
	
	% A promising  approach is the measurement of ultrasonic velocity
	%**************************************************************
	\section{Motivation}
	\lipsum[2-4]
	
	
	\section{Project Objectives}
	This project investigates......
	\begin{itemize}
		\item objective 1
		\item objective 2
		\item objective 3
		\item objective 4
		\item  objective 5
		
	\end{itemize}
	
	
	
	
	
	%***************************************************************
	\section{Literature Review}
	In this project, we propose methods .................. A brief description of the contributions of this thesis is given below:
	\begin{itemize}
		\item
		
		
	\end{itemize}
	%****************************************************************
	%\section{Conclusions}
	
	%****************************************************************
	\newpage
	\section{Methodology}
	The project consists  of seven  chapters, and the organization of the project is as follows:
	\begin{enumerate}
		\item
		
	\end{enumerate}
	%\pagebreak
	%****************************************************************
	
	\newpage
	\section{Published Article Review}
	\begin{enumerate}
		\item Publication Name 1  
		\item publication 2
		\item publication 3
		\item publication 4
		\item publication 5
		
	\end{enumerate}
	%%%%%%%%%%%%%%%%%%%%%%%%%%%%%%%%%%%%%%%%%%%%%%%%%
	
	\newpage
	\section{Conclusion}
	Write the conclusion of the project.
	\begin{enumerate}
		\item
	\end{enumerate}
	
	\newpage
	\section{Future Work}
	Write the future scopes of the project.
	\begin{enumerate}
		\item
		
	\end{enumerate}
	\bibliographystyle{unsrt}
	\bibliography{ref}
	%\nocite{*}
	
\end{document}